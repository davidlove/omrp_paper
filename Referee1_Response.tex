\documentclass[11pt,notitlepage,onecolumn]{article}
%%%%%%%%%%%%%%%%%%%%%%%%%%%%%%%%%%%%%%%%%%%%%%%%%%%%%%%%%%%%%%%%%%%%%%%%%%%%%%%%%%%%%%%%%%%%%%%%%%%%%%%%%%%%%%%%%%%%%%%%%%%%%%%%%%%%%%%%%%%%%%%%%%%%%%%%%%%%%%%%%%%%%%%%%%%%%%%%%%%%%%%%%%%%%%%%%%%%%%%%%%%%%%%%%%%%%%%%%%%%%%%%%%%%%%%%%%%%%%%%%%%%%%%%%%%%
\usepackage{makeidx}
\usepackage{amsfonts}
\usepackage{amsmath}
\usepackage{amssymb}
\usepackage{chicago}
\usepackage[singlespacing]{setspace}

\setcounter{MaxMatrixCols}{10}


\setlength{\textheight}{9in}
\setlength{\textwidth}{6.5in}
\setlength{\topmargin}{-0.5in}
\setlength{\oddsidemargin}{0.0in}

\newcommand{\noi}{\noindent}

% Definitions for overlapping batches
\newcommand{\gb}{\bar{G}}
\newcommand{\gbb}{\bar{\gb}}
\newcommand{\db}{\bar{D}}
\newcommand{\dbb}{\bar{\db}}


\begin{document}

%%%%%%%%%%%%%%%%%%%%%%%%%%%%%%%%%%%%%%%%%%%%%%%%%%%%%%%%%%%%%%%%%%%%%%%%%%%

\singlespacing

\baselineskip0.26in

%%%%%%%%%%%%%%%%%%%%%%%%%%%%%%%%%%%%%%%%%%%%%%%%%%%%%%%%%%%%%%%%%%%%%%%%%%%

\pagebreak

\begin{center}
\textbf{\Large Response to Referee 1} 
\medskip

{\large Overlapping Batches for the Assessment of Solution Quality in Stochastic Programs}
\medskip

{\footnotesize by D. Love and G. Bayraksan}
\end{center}

\bigskip

\bigskip

%%%%%%%%%%%%%%%%%%%%%%%%%%%%%%%%%%%%%%%%%%%%%%%%%%%%%%%%%%%%%%%%%%%%%%%%%%%

We thank the referee for the helpful comments and suggestions. 
We incorporated all of the recommended changes and believe the paper is improved as a result. 
Below, we go through in detail these changes. 
\medskip

\bigskip 

%%%%%%%%%%%%%%%%%
%% Need to be careful with the minor changes. Will these be affected?
%% Need to CHECK AT THE VERY END... FOR ANY CHANGES on TOP OF these...
%%    - GB
%%%%%%%%%%%%%%%%%

\noi  
{\large \bf ``Major'' comments:}
\medskip 

\begin{itemize}
\item[\textbf{M1.}] \textit{In the typical simulation setting, the computational effort required to produce the output dominates, and hence one would like to ``make the most'' of the output in terms of analysis. 
Sometimes, a similar motivation can arise in stochastic optimization because there may be a computationally expensive underlying simulation model that must be run to produce the i.i.d.\ realizations that are then used in an SAA approach. 
(While they don't now, the authors could include this as motivation for
their work.)}
\end{itemize}

\noi xxx 
We thank the referee for this suggestion. 
You are right.
There are cases when generating scenarios is expensive in the stochastic optimization setting as well. 
We changed the introduction based on your, the associate editor and the other reviewer's comments. 
We specifically added this as a motivation. 
\medskip

\begin{itemize}
\item[] \textit{However, perhaps more often, sampling the i.i.d. realizations is computationally cheap compared to solving the resulting SAA instances in stochastic optimization. 
In this latter case, we have a choice between simply increasing the number of batches in a non-overlapping method or using the overlapping batch means approach (with some degree of overlap) in order to reduce variance.
Based on the paper's results, perhaps the computational effort required to achieve a level of variance reduction should guide this choice. 
The authors do point to the fact that warm starts in a stochastic
optimization algorithm can be exploited much more easily when using overlapping batch means than when not using overlapping batches. 
Some attempt to explore this tradeoff would be valuable. 
The computational results use sorting (for the news vendor problem), a dynamic programming algorithm (for a stochastic knapsack model), and a decomposition method (for two-stage stochastic linear programs). 
While I am not suggesting that this exploration needs to be made for all three types of algorithms, even exploring it in the case of the sorting algorithm (where some algorithms are well suited for almost-sorted lists) could provide valuable insight.}
\end{itemize}

\noi 
Based on your and the associate editor's comments, we studied the computational efficiency of the OMRP estimators. 
The results of this study are presented in a new section, Section 5.4 (Computational Efficiency).
%We studied the computational efficiency of the OMRP estimators, based on your and the associate editor's comments, and provided a summary of our results in a new section, Section 5.4 (Computational Efficiency). 
We first approached efficiency by the reciprocal of variance multiplied by the computation time, per associate editor's suggestion. 
Maximizing the computational efficiency defined in this manner provided guidelines on selecting the degree of overlap. 
We then looked at how OMRP compared to increasing the number batches in MRP.
We examined computational efficiency both theoretically and empirically. 
For empirical evaluation, we used the newsvendor problem with insertion sort, based on your suggestions. 
We found that when effective warm-starting is available, overlapping can lead to more efficient estimators than (nonverlapping) MRP, both when the direct nonoverlapping counterpart is used or when the number of batches in MRP is increased to match the variance of the overlapping counterpart.  
All these results are presented in Section 5.4 (Computational Efficiency). 
Thank you for this suggestion. 
We believe the study of computational efficiency significantly improved the paper. 
\medskip



\begin{itemize}
\item[\textbf{M2.}] \textit{The choice of the lower bound that is embedded in equation (8) is interesting. 
(The upper bound in equation (8) is completely natural as the sample mean of the $n_l$ realizations of the function evaluated at $\hat{x}$.) 
It seems that there are also other natural choices that could be made here for the lower bound. 
There is some discussion of this choice in the text that follows equations (7)-(9), but can the authors say more? For example, do the consistency and asymptotic validity results that follow hinge on this choice? Or, do those results also hold for other choices, like forming $\gbb$ as a sample
mean of the underlying $\gb$ values. 
We know that the individual gap estimators in (7) are nonnegative, and if we formed $\gbb$ as a sample mean of $\gb$ values, it would be nonnegative. 
Is the $\gbb$, as specified in (8), nonnegative? 
Have the authors computationally explored other choices here? 
Any insight in this vein would be valuable to the reader in understanding the choice made in equation (8).}
\end{itemize}

\noi
You are right; we did not clearly motivate this choice of $\gbb$.  
We added explanations after the definition of the OMRP estimators, motivating this choice and relating it to the theoretical results presented in the paper.\smallskip 


\noi 
As you pointed out for the upper bound in (8), this $\gbb$ was chosen for each observation to have equal weight when the sampling problems yield optimal solutions $x_j^* = x^*, \ \forall j$. 
This fact is used in Lemmas 1 and 2, which are then used---directly or indirectly---in proving Theorems 1--3. 
When the alternative $\gbb$, formed as a sample mean of $\gb$ values, is used, theoretical results presented in the paper can be recovered by appropriately adjusting the coefficients of the $\dbb$ terms.
Notice that these results establish convergence of  $\gbb$, $VG$, etc.\ to their nonoptimized counterparts $\dbb$, $VD$, etc.
When the coefficients of $\dbb$ are adjusted, however, $\dbb$ and $VD$ are {\it not exactly} the same as the overlapping estimators of simulation output analysis. 
Because our results depend on the asymptotic properties of the overlapping estimators already established in the simulation literature, we did not want to make any changes. 
It may be possible to recover the same results but we did not pursue it here. 
We actually performed computations using $\gbb$ formed as a sample mean of $\gb$ values and the results were essentially the same. 
However, due to subtle theoretical reasons explained above, we decided not to include or discuss this choice. 
\medskip 

\bigskip 

\noi {\large \bf Minor comments:}

\bigskip



\begin{itemize}
\item[1.] \textit{p1, l49: Suggest: The Overlapping \ldots }
\end{itemize}

\noindent 
Done. 
\medskip 

\begin{itemize}
\item[2.] \textit{p2, l16, l22: overlapping is misspelled twice.}
\end{itemize}

\noindent 
Corrected. Thanks for catching these typos.
\medskip 


\begin{itemize}
\item[3.] \textit{p3, l23: Do the authors mean ``relaxations'' rather than ``relations''?}
\end{itemize}

\noindent 
Yes. Corrected. 
\medskip 


\begin{itemize}
\item[4.] \textit{p3, l38: Suggest: estimator of the optimality gap \ldots}
\end{itemize}

\noindent 
Done. 
\medskip 


\begin{itemize}
\item[5.] \textit{p4, l34: Suggest: once the data are generated through a simulation, they \ldots}
\end{itemize}

\noindent 
Done. 
%Please note that this sentence has been changed slightly based on the comments of the other referee. 
\medskip 


\begin{itemize}
\item[6.] \textit{p4, l42: Suggest: Partial overlap results in fewer batches, hence fewer optimization \ldots}
\end{itemize}

\noindent 
We reworded this sentence to read as ``Partial overlapping results in a smaller number of batches; hence, fewer optimization \ldots '' based on the comments of the other referee.
\medskip 


\begin{itemize}
\item[7.] \textit{p6, l32: Suggest: The degrees \ldots make \ldots}
\end{itemize}

\noindent 
Done. 
% This sentence might be deleted or changed to the denominator term, etc. CHECK! 
\medskip 


\begin{itemize}
\item[8.] \textit{p7, l18: While a confidence level of $1-\alpha$ may be consistent with a level of significance of $\alpha$, perhaps this could be reworded (as it seemed to distract this reader a bit).}
\end{itemize}

\noindent  
Changed the wording here. Thank you. 
\medskip 


\begin{itemize}
\item[9.] \textit{p10, l23: Suggest: convergence of the sampled solution \ldots}
\end{itemize}

\noindent 
Done. 
\medskip 


\begin{itemize}
\item[10.] \textit{p12, l53: Suggest: Strong consistency \ldots has been \ldots}
\end{itemize}

\noindent 
Done. 
\medskip 


\begin{itemize}
\item[11.] \textit{p14, l13-14: Suggest: The fringe batches come \ldots The first and last batch are identically \ldots}
\end{itemize}

\noindent 
Done. 
\medskip 


\begin{itemize}
\item[12.] \textit{ p14, l23: Please add $j = 1$ to the sum.}
\end{itemize}

\noindent 
Added. Thanks for catching this.
\medskip 


\begin{itemize}
\item[13.] \textit{p14, l35: The authors have gone to some length to distinguish different types of $A_j$ terms.
However, this equation has an $\mathbb{E}A_j^2$ term that has been collapsed from a sum, but we don't know ``which $j$''. 
Should there be a $\max_j \mathbb{E}A_j^2$ or some other treatment here?}
\end{itemize}

\noindent xxx NEED TO CHECK AND CORRECT AS NECESSARY. 
\medskip 


\begin{itemize}
\item[14.] \textit{p15, Theorem 3: A hypothesis in the theorem is that $\hat{x} \neq x^*$, and yet the proof starts with the case that $\hat{x} = x^*$. 
That hypothesis could be removed if the claim had a ``$\geq$'' or this statement could be removed or modified.}
\end{itemize}

\noindent Thank you for pointing this out.  
We removed this statement from the proof of Theorem 3 and added it as a remark to the paragraph after the theorem.
\medskip 



\begin{itemize}
\item[15.] \textit{p17, l44: Suggest: A summary of results \ldots experiments is depicted \ldots}
\end{itemize}

\noindent Done.
%\ \ \ XXXX CHECK LATER: THIS SECTION MIGHT CHANGE. 
\medskip 


\begin{itemize}
\item[16.] \textit{p17, l55: Perhaps add a comment on why DB1 may not experience the asymptotic variance reduction? 
(Is it because it may lack a unique solution?)}
\end{itemize}

\noindent
Done.
% xxx Yes, it is partially because we are not sure if it has a unique optimal solution as DB1 has never been solved before. 
%Even if it has been solved, the variance reduction is shown asymptotically but the computations are only for small sample sizes. 
%We reworded this to reflect these points.
%Thank you for this suggestion. 
\medskip 


\begin{itemize}
\item[17.] \textit{p18, figures: The figures are a bit small and hence can be hard to read without enlarging them (but perhaps this will be taken care of at production).}
\end{itemize}

\noindent 
We split these two figures into three and enlarged each graph. 
We also enlarged the legends in these graphs to improve readability.
All the new graphs added to the paper in Section 5.4 (Computational Efficiency) are also larger, with legible legends. 
\medskip 


\begin{itemize}
\item[18.] \textit{p18, l57: The authors speak of coverage probabilities for DB1, but how can this be done without knowing an optimal solution to DB1?}
\end{itemize}

\noindent 
We used a Latin hypercube sample of size 50,000 estimate $\mu_{\hat{x}}$ for DB1, which was then used to estimate the coverage probability of the OMRP confidence intervals for this test problem.
We explain this at the end of Section 5.1 (Test Problems). 
We also changed the wording in Section 5.3 (Results of Experiments) to make it clear that the coverage probability is estimated for DB1. 
\medskip 


\begin{itemize}
\item[19.] \textit{p19: It may be possible to expand on the last paragraph of Section 5, in giving guidance on selecting the degree of overlap, depending on the authors' response to comment M1 above.}
\end{itemize}

\noindent 
Thank you for this suggestion. 
Based on our study of computational efficiency of the OMRP estimators, we  provided an updated and expanded guidance on selecting the degree of overlap. 
This guidance appears in the last paragraph of \S 5.4 (Computational Efficiency). 

\end{document}
