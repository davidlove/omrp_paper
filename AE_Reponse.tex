\documentclass[11pt,notitlepage,onecolumn]{article}
%%%%%%%%%%%%%%%%%%%%%%%%%%%%%%%%%%%%%%%%%%%%%%%%%%%%%%%%%%%%%%%%%%%%%%%%%%%%%%%%%%%%%%%%%%%%%%%%%%%%%%%%%%%%%%%%%%%%%%%%%%%%%%%%%%%%%%%%%%%%%%%%%%%%%%%%%%%%%%%%%%%%%%%%%%%%%%%%%%%%%%%%%%%%%%%%%%%%%%%%%%%%%%%%%%%%%%%%%%%%%%%%%%%%%%%%%%%%%%%%%%%%%%%%%%%%
\usepackage{makeidx}
\usepackage{amsfonts}
\usepackage{amsmath}
\usepackage{amssymb}
\usepackage{chicago}
\usepackage[singlespacing]{setspace}

\setcounter{MaxMatrixCols}{10}


\setlength{\textheight}{9in}
\setlength{\textwidth}{6.5in}
\setlength{\topmargin}{-0.5in}
\setlength{\oddsidemargin}{0.0in}

\newcommand{\noi}{\noindent}
\begin{document}


%%%%%%%%%%%%%%%%%%%%%%%%%%%%%%%%%%%%%%%%%%%%%%%%%%%%%%%%%%%%%%%%%%%%%%%%%%%

\singlespacing

\baselineskip0.26in

%%%%%%%%%%%%%%%%%%%%%%%%%%%%%%%%%%%%%%%%%%%%%%%%%%%%%%%%%%%%%%%%%%%%%%%%%%%

\pagebreak

\begin{center}
\textbf{\Large Response to the Associate Editor} 
\medskip

{\large Overlapping Batches for the Assessment of Solution Quality in Stochastic Programs}
\medskip

{\footnotesize by D. Love and G. Bayraksan}
\end{center}

\bigskip

\bigskip 


%%%%%%%%%%%%%%%%%%%%%%%%%%%%%%%%%%%%%%%%%%%%%%%%%%%%%%%%%%%%%%%%%%%%%%%%%%%

\noi
We thank the Associate Editor for the helpful comments and suggestions. 
We incorporated all of the recommended changes and believe the paper is improved as a result. 
Below, we go through in detail the suggestions of the Associate Editor.
\medskip


\begin{itemize}
\item[]\textit{The paper is not clear in laying out a publishable contribution, and the motivation of the work is not clear.}
\end{itemize}

\noi
We revised the introduction to clearly reflect the contributions of the paper and the motivation for the work. 
Please see page four of the revised manuscript. 
This section also includes contributions based on the new material on computational efficiency added to the paper (contribution (iv)) and a comparison with the proceedings version of the paper (based on the editor-in-chief's suggestion).\medskip  

\begin{itemize}
\item[] 
%\textit{
%The experiments show that the confidence interval average widths and observed coverage probabilities are the same with overlapping batch means; the only difference is in decreased variability of the confidence interval widths. 
%That is, the variance of the variance estimator is reduced. 
%This second-order benefit should be weighed against the increased computation cost (not to mention complexity and restricted applicability) of the overlapping batch means method.
%}
\textit{\ldots The issue of computational efficiency is not addressed  in the paper. 
In this setting the efficiency is often defined as the reciprocal of the computation time multiplied by the variance. 
If this quantity were compared (overlapping batch means versus batch means) in the numerical experiments, I am curious if there are any examples where the efficiency is improved.}
\end{itemize}

\noi 
Thank you for this suggestion. 
It is indeed important to study computational efficiency in this context. 
We created a new section, \S 5.4 (Computational Efficiency), to study the computational efficiency of overlapping estimators in the stochastic optimization setting. 
%When an algorithm has warm starting capability, as the degree of overlapping increases,  solution time per batch could decrease; as a result, computational efficiency gains could be observed.  
We discussed efficiency for a given functional form of the solution time per batch. 
%Depending on the effectiveness of warm starting, maximizing computational efficiency provides guidelines on selecting the degree of overlapping. 
%\medskip 
In addition, based on the comments of Referee 1, we studied efficiency of overlapping estimators compared to increasing the number of batches in MRP to obtain (nonoverlapping) estimators with equivalently lower variances. 
Our results indicate that when warm starting is effective, overlapping is computationally more efficient.
Even in cases where the effectiveness of warm starting is quite low, computational efficiency can be observed with a low degree of overlap (e.g., when approximately 10\% of observations are overlapped).\bigskip 

\noi 
We illustrated these results on the newsvendor problem using insertion sort. 
Our experiments show that (i) the functional form used provides a good representation, (ii) increased computational efficiency over the nonoverlapping counterpart is indeed observed over a range of overlapping levels, and (iii) this analysis provides a way to select the degree of overlapping by maximizing efficiency.\bigskip 
 
\noi
Finally, we provided guidelines on selecting the degree of overlap at the end of \S 5.4 (Computational Efficiency) based on our study. 
We believe the paper has improved as a result of the study of computational efficiency. 
Thank you for this suggestion. 
\medskip 


\begin{itemize}
\item[] \textit{The authors could argue the publishability of the work by establishing improved efficiency for some examples, or providing some other argument for the usefulness of the results.  
A link between less variable confidence interval widths and improved stochastic optimization efficiency would be helpful.}
\end{itemize}

\noi 
We established improved computational efficiency on the newsvendor problem using insertion sort.
We also (i) changed \S 1 (Introduction) to clearly motivate the study and argue usefulness of the results, paying specific attention to efficiency gains; (ii) updated \S 6 (Conclusions) to summarize these new results, again paying attention to computational efficiency; (iii) provided a study of computational efficiency gains of overlapping in the context of assessing solution quality in stochastic programming in the new \S 5.4 (Computational Efficiency); and finally (iv) updated the abstract.  
%(Please see our response to the previous comment.)
\medskip



\bigskip 


\noi {\bf \large Minor comments:}
\medskip 



\begin{itemize}
\item[] \textit{2.38. The meaning of the set $X$ being independent of the random variable $\xi$ is not clear to me.}
\end{itemize}

\noi 
We meant to say that the feasible region only consists of deterministic constraints and does not have any stochastic constraints such as probabilistic or expected-value constraints. 
We changed the wording to reflect this. 
\medskip 


\begin{itemize}
\item[] \textit{4.24. \ldots data are generated \ldots}
\end{itemize}

\noi 
Done.
\medskip

\noi
{\it Note:}
We moved this sentence from \S 1 (Introduction)---to clearly state the contributions of the paper---to \S 3 (Overlapping Multiple Replications Procedure), where we provided more technical details. 
\medskip 

\begin{itemize}
\item[] \textit{19.11 \ldots such a scheme \ldots}
\end{itemize}

\noi
This sentence was removed as a result of the computational efficiency study. 
Based on the examination of computational efficiency, we provided new guidelines on selecting the degree of overlapping. 


\end{document}
