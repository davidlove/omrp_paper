\documentclass[11pt,notitlepage,onecolumn]{article}
%%%%%%%%%%%%%%%%%%%%%%%%%%%%%%%%%%%%%%%%%%%%%%%%%%%%%%%%%%%%%%%%%%%%%%%%%%%%%%%%%%%%%%%%%%%%%%%%%%%%%%%%%%%%%%%%%%%%%%%%%%%%%%%%%%%%%%%%%%%%%%%%%%%%%%%%%%%%%%%%%%%%%%%%%%%%%%%%%%%%%%%%%%%%%%%%%%%%%%%%%%%%%%%%%%%%%%%%%%%%%%%%%%%%%%%%%%%%%%%%%%%%%%%%%%%%
\usepackage{makeidx}
\usepackage{amsfonts}
\usepackage{amsmath}
\usepackage{amssymb}
\usepackage{chicago}
\usepackage[singlespacing]{setspace}

\setcounter{MaxMatrixCols}{10}


\setlength{\textheight}{9in}
\setlength{\textwidth}{6.5in}
\setlength{\topmargin}{-0.5in}
\setlength{\oddsidemargin}{0.0in}

\newcommand{\noi}{\noindent}

% Definitions for overlapping batches
\newcommand{\gb}{\bar{G}}
\newcommand{\gbb}{\bar{\gb}}
\newcommand{\db}{\bar{D}}
\newcommand{\dbb}{\bar{\db}}


\begin{document}

%%%%%%%%%%%%%%%%%%%%%%%%%%%%%%%%%%%%%%%%%%%%%%%%%%%%%%%%%%%%%%%%%%%%%%%%%%%

\singlespacing

\baselineskip0.26in

%%%%%%%%%%%%%%%%%%%%%%%%%%%%%%%%%%%%%%%%%%%%%%%%%%%%%%%%%%%%%%%%%%%%%%%%%%%

\pagebreak

\begin{center}
\textbf{\Large Response to Reviewer 1} 
\medskip

on the revision of\\ 


{\large Overlapping Batches for the Assessment of Solution Quality in Stochastic Programs}
\medskip

{\footnotesize by D. Love and G. Bayraksan}
\end{center}

\bigskip


%%%%%%%%%%%%%%%%%%%%%%%%%%%%%%%%%%%%%%%%%%%%%%%%%%%%%%%%%%%%%%%%%%%%%%%%%%%
\noi
We thank the Reviewer for the careful read and the helpful suggestions. 
We incorporated all of the recommended changes. 
Below, we go through in detail these changes. 
\medskip

\bigskip 

\begin{itemize}
\item \textit{p6, l47: The sentence with ``are no longer independent'' doesn't quite make sense, given the context introduced so far in which the observations of y can be dependent.}
\end{itemize}

\noi 
Deleted this sentence. 
Thank you; good catch! 
We revised and rearranged the beginning of this paragraph for better flow.  
\medskip

\begin{itemize}
\item \textit{p11, l16: Delete extra space in \S4.4.}
\end{itemize}

\noi 
Done. 
\medskip



\begin{itemize}
\item \textit{p13, l21-22: I know what the authors mean, but I can't quite parse the last sentence of the proof of Lemma 2.}
\end{itemize}

\noi
Deleted this sentence. Shortened the paragraph as suggested by Reviewer 2. 
\medskip 


\begin{itemize}
\item \textit{p13, l37: Suggest: is nearly identical to that above.}
\end{itemize}

\noindent 
Deleted this sentence as suggested by Reviewer 2. 
\medskip 


\begin{itemize}
\item \textit{p17, l58: ``solving'' could be deleted so we don't have the notion of ``solving ... runs''; this may appear elsewhere, too.}
\end{itemize}

\noindent 
Done.
\medskip 

\end{document}
