\documentclass[11pt,notitlepage,onecolumn]{article}
%%%%%%%%%%%%%%%%%%%%%%%%%%%%%%%%%%%%%%%%%%%%%%%%%%%%%%%%%%%%%%%%%%%%%%%%%%%%%%%%%%%%%%%%%%%%%%%%%%%%%%%%%%%%%%%%%%%%%%%%%%%%%%%%%%%%%%%%%%%%%%%%%%%%%%%%%%%%%%%%%%%%%%%%%%%%%%%%%%%%%%%%%%%%%%%%%%%%%%%%%%%%%%%%%%%%%%%%%%%%%%%%%%%%%%%%%%%%%%%%%%%%%%%%%%%%
\usepackage{makeidx}
\usepackage{amsfonts}
\usepackage{amsmath}
\usepackage{amssymb}
\usepackage{chicago}
\usepackage[singlespacing]{setspace}

\setcounter{MaxMatrixCols}{10}

\setlength{\textheight}{9in}
\setlength{\textwidth}{6.5in}
\setlength{\topmargin}{-0.5in}
\setlength{\oddsidemargin}{0.0in}

\newcommand{\noi}{\noindent}

\begin{document}

%%%%%%%%%%%%%%%%%%%%%%%%%%%%%%%%%%%%%%%%%%%%%%%%%%%%%%%%%%%%%%%%%%%%%%%%%%%

\singlespacing

\baselineskip0.26in

%%%%%%%%%%%%%%%%%%%%%%%%%%%%%%%%%%%%%%%%%%%%%%%%%%%%%%%%%%%%%%%%%%%%%%%%%%%

\pagebreak

\
\begin{center}
\textbf{\Large Response to Referee 2} 
\medskip

{\large Overlapping Batches for the Assessment of Solution Quality in Stochastic Programs}
\medskip

{\footnotesize by D. Love and G. Bayraksan}
\end{center}

\bigskip


%%%%%%%%%%%%%%%%%%%%%%%%%%%%%%%%%%%%%%%%%%%%%%%%%%%%%%%%%%%%%%%%%%%%%%%%%%%

We thank the referee for the helpful comments and suggestions. 
We incorporated all of the recommended changes and believe the paper is improved as a result. 
Below, we go through in detail these changes. 
\medskip

\bigskip 

%%%%%%%%%%%%%%%%%
%% Need to be careful with the minor changes. Will these be affected?
%% Need to CHECK AT THE VERY END... FOR ANY CHANGES on TOP OF these...
%%    - GB
%%%%%%%%%%%%%%%%%

\noi  
{\large \bf General Comments:}
\medskip 


\begin{itemize}
\item[1.] \textit{The main contributions of the paper need to be stated more clearly and succinctly.}
\end{itemize}

\noindent 
xxx  We agree. 
We changed the introduction to clearly state the contributions of the paper and \ldots  and the presentation of the paper significantly improved.  Thank you.
\medskip 


\begin{itemize}
\item[2.] \textit{It is not necessary to go into such detail in the OBM review.}
\end{itemize}

\smallskip 

\noindent 
To shorten and streamline the OBM review, and to introduce the main results used in the paper, we combined the previous two sections---\S 2.1 (Overlapping Batch Means) and \S 3.1 (Variably Overlapping Batches)---with other explanations on OBM found later in the paper. 
The combined review appears in the background section, \S 2.1 (Batch Means). 
To improve readability, we decided to create two subsections: \S 2.1.1 (Nonoverlapping Batch Means) and \S 2.1.2 (Overlapping Batch Means). 
While we removed a number of equations and shortened the text, we kept some explanations that are necessary for a reader who is not familiar with OBM (e.g., readers with an optimization focus) to follow the main ideas. 
These basic explanations appear, for instance, in the first paragraph of \S 2.1.2 (Overlapping Batch Means).  
We hope to be able to reach a broader audience this way. 
As a result of these changes, 
\begin{itemize}
	\item[(i)]  instead of former equations (2) and (5), both on overlapping batch means variance estimator, we have only one equation (2); 
	
	\item[(ii)]  the confidence interval estimation using OBM is discussed only once;

	\item[(iii)] the former equation (3) is removed;

	\item[(iv)] the discussion that appeared at the beginning of \S 4.4 (previously: Variance Reduction, now: Asymptotically Lower Variances) is moved to this review section. 
\end{itemize}

\noi
With these changes, the OBM review became more streamlined, focusing on the main results used later in the paper. 
Thank you.
\bigskip 

\begin{itemize}
\item[3.] \textit{The English needs to be tightened up somewhat.}
\end{itemize}

\noindent  
xxxx
You are right. 
We went through the paper and tightened up the prose. 
We also incorporated all the changes you suggested below and hope the exposition of the paper is improved as a result.
We hope that any remaining English issues could be corrected by the English editor of the journal. (IS THERE ONE????) 
xxxx
\medskip 

\bigskip 


\noi  
{\large \bf Specific Comments:}
\medskip 


\begin{itemize}
\item[] \textit{Page 1, Line 32 (and elsewhere): I believe that you ``apply the OBM method to stochastic programming'' rather than extend it.}
\end{itemize}

\noi
Changed.  
We also went through the paper to correct other occurrences. 
In particular, this was also used in Section 6 (Conclusions). 
We changed the first sentence of Section 6 (Conclusions) to use the word `apply' instead of `extend'. 

\medskip 



\begin{itemize}
\item[] \textit{1, 35 (and elsewhere): ``We give conditions'' instead of ``We show conditions''}
\end{itemize}

\noi
Changed. 
We also went through the paper and corrected other occurrences. 
The current version of the paper either has `give conditions' or `provide conditions'. 
\medskip 


\begin{itemize}
\item[] \textit{1, 37: What are the classical results? Reference?}
\end{itemize}

\noi
We reworded this sentence to improve clarity and provided references.   
\medskip 



\begin{itemize}
\item[] \textit{1, 49: Suggested rewrite to reduce ambiguity: ``The overlapping batch means (OBM) \ldots to obtain variance estimators having variances that are comparatively lower than their nonoverlapped counterparts. 
In the context of steady-state simulation, OBM is often used to estimate the variance of the sample mean, which is itself an estimator for the mean. 
Schmeiser et al.\ [1990] use overlapping estimators to estimate the
variance of estimators for non-means. 
More recently, overlapping estimators based on standardized time series (as opposed to batch means) have been evaluated [Alexopoulos et al., 2007a,b].'' 
In this rewrite, note the use of active and passive referencing formats.}
\end{itemize}

\noi
Thank you for this suggestion of rewrite and for your careful and detailed review of the paper. 
We rewrote this paragraph based on your corrections.
We also went through the paper and made changes to the language, paying specific attention to active and passive referencing formats. 
\medskip 


\begin{itemize}
\item[] \textit{2, 5: ``never before been studied . . . context.''}
\end{itemize}

\noi
Changed.
\medskip 


\begin{itemize}
\item[] \textit{2, 7: By ``same benefits'', you mean moderate bias and comparatively low variance, right?}
\end{itemize}

\noi
xxx  
\medskip 


\begin{itemize}
\item[] \textit{2, 8: Start a new paragraph with ``A natural setting\ldots ''.}
\end{itemize}

\noi
Done.
\medskip 


\begin{itemize}
\item[] \textit{2, 10: ``especially'' instead of ``esp.''. 
Also, ``This'' instead of ``Assessing solution quality in stochastic programming'' (since it's repetitive)}
\end{itemize}

\noi
Both done.  
\medskip 


\begin{itemize}
\item[] \textit{2, 13: ``Because optimality gap''}
\end{itemize}

\noi
Done.
\medskip 


\begin{itemize}
\item[] \textit{2, 24: ``under which the resulting gap estimators exhibit reduced variance when compared to their counterparts obtained via nonoverlapping batches.''}
\end{itemize}

\noi
Done.
\medskip 


\begin{itemize}
\item[] \textit{2, 24 (and elsewhere): Note: I would be careful about the use of the term ``variance reduction'', which usually refers to the class of techniques such as common random numbers, antithetics, importance sampling, etc. 
I view what you're doing in this paper as simply proposing a new estimator whose variance is lower than that of a competing estimator.}
\end{itemize}

\noi
xxx  You are right. 
Thanks for pointing this out. We corrected the wording throughout the paper. 
\medskip 


\begin{itemize}
\item[] \textit{2, 27: ``and that the interval \ldots valid. 
We also present''}
\end{itemize}

\noi
Done. 
\medskip 


\begin{itemize}
\item[] \textit{2, 28: ``Our experiments indicate that the use of overlapping estimators results in gap estimators achieving lower variance than their nonoverlapping counterparts, along with comparable bias and interval coverage --- for both small- and large-sample cases.}
\end{itemize}

\noi
Done.
\medskip 


\begin{itemize}
\item[] \textit{2, 34 (and elsewhere): Might want to boldface all vectors.}
\end{itemize}

\noi
Done.  
\medskip 


\begin{itemize}
\item[] \textit{2, 42 (and elsewhere): ``more-restrictive moment'' (hyphenate certain non-adverbial compound adjectives)}
\end{itemize}

\noi
xxx  Done. But I am not sure if it is correct this way.  
Thank you; we went through the paper and hyphenated all the compound adjectives.
\medskip 


\begin{itemize}
\item[] \textit{2, 51: You can delete ``(SAA)'', since the notation isn't used again in the paper.}
\end{itemize}

\noi
Done.  
\medskip 


\begin{itemize}
\item[] \textit{3, 11: ``solution; and''}
\end{itemize}

\noi
Done.
\medskip 


\begin{itemize}
\item[] \textit{3, 24 (and elsewhere): ``Lagrangian, or'' (comma after penultimate entry in a conjunctive list)}
\end{itemize}

\noi
Done.  
We went through the paper and added commas before `and' and `or' in a list of items with three or more items.
\medskip 


\begin{itemize}
\item[] \textit{3, 37: ``we can simply use $z_m^*$''}
\end{itemize}

\noi
Done. 
\medskip 


\begin{itemize}
\item[] \textit{3, 38: ``estimator of the optimality''}
\end{itemize}

\noi
Done.  
\medskip 


\begin{itemize}
\item[] \textit{3, 53: ``The sample variance of these estimators is used to form confidence intervals for the optimality gap (we further review MRP in \S 2.2).''}
\end{itemize}

\noi
Done. 
\medskip 


\begin{itemize}
\item[] \textit{4, 5 (and elsewhere): ``see, e.g., Bertocchi et al. [2000]; \ldots ''}
\end{itemize}

\noi
We put a comma after `see'; however, we still kept the same referencing style `[Bertocchi et al., 2000; \ldots]' as this sentence refers to the papers and not directly to the authors. 
We also went through the paper to make sure all `see, e.g.,' have the appropriate commas.
\medskip 


\begin{itemize}
\item[] \textit{4, 14: Delete the sentence ``Especially \ldots '' since it doesn't really add anything to the discussion.}
\end{itemize}

\noi
Deleted.
\medskip 


\begin{itemize}
\item[] \textit{4, 17: ``interest is that of estimating''}
\end{itemize}

\noi
Done.  
\medskip 


\begin{itemize}
\item[] \textit{4, 20--27: Even though I believe everything, these lines are a bit awkward and confusing.}
\end{itemize}

\noi
xxx   DAVID-- NEED YOUR HELP with this. 
I don't know how to reword it to make it less awkward.
\medskip 


\begin{itemize}
\item[] \textit{4, 31: What are the desired statistical properties?}
\end{itemize}

\noi
Thanks for pointing out the ambiguity here. 
We reworded this sentence and broke it into two sentences to make it clear. 
\medskip 


\begin{itemize}
\item[] \textit{4, 33: ``Another difference is that''. By the way, what is ``difference'' referring to?}
\end{itemize}

\noi
Done.
We also changed the sentence to make it clear what the ``difference'' is referring to. 
\medskip 


\begin{itemize}
\item[] \textit{4, 36: ``due to the need to solve sampling problem"}
\end{itemize}

\noi
Done.  
\medskip 


\begin{itemize}
\item[] \textit{4, 42: ``Partial overlapping results in a smaller number of batches; hence, fewer optimization''}
\end{itemize}

\noi
Done.  
\medskip 


\begin{itemize}
\item[] \textit{4, 45: The sentence ``Second, \ldots '' needs some clarification, since it states a conclusion that hasn't so far been justified.}
\end{itemize}

\noi
xxx  
\medskip 


\begin{itemize}
\item[] \textit{4, 51: ``discuss overlapping''}
\end{itemize}

\noi
Done.  
\medskip 


\begin{itemize}
\item[] \textit{4, 54 (and elsewhere): When you refer to ``variable amounts of overlap'', that is really nothing new. Certainly, any overlapping estimator will have end effects in which the overlap amounts differ from those in the ``middle'' region of the data.}
\end{itemize}

\noi
%You are right.
We meant to say that the degree of overlap is parametrized and this parameter can be chosen by the user. 
We reworded this sentence to reflect this. 
We also went through the paper, removed other occurrences, and avoided using this term in the new text added. 
\medskip 



\begin{itemize}
\item[] \textit{4, 57 (and elsewhere): ``classical case'' refers to nonoverlapping batch means, right?}
\end{itemize}

\noi
No; by `classical case', we meant the use of overlapping batches in simulation output analysis (as opposed to the assessment of solution quality in stochastic programming). 
We reworded this sentence to make it clear. 
Please note that based on your earlier comments, we reworded this sentence to avoid using the term `variance reduction'. 
We also went through the paper and removed several uses of the word `classical' that referred to nonoverlapping batches to reduce ambiguity.
%(as these were referring to nonoverlapping batches). 
%In the current version of the paper, `classical case' is used to only indicate  the use of overlapping batches in simulation output analysis. 
\medskip 



\begin{itemize}
\item[] \textit{5, 12: Does $\sigma^2$ refer to the marginal variance or the ``variance parameter'' (sum of the covariances at all lags)?}
\end{itemize}

\noi
Variance parameter. 
We revised this sentence and added a footnote defining the variance parameter for readers who might not be familiar with this term (e.g., readers with an optimization focus). 
\medskip 



\begin{itemize}
\item[] \textit{5, 14: ``realization of the univariate stochastic''}
\end{itemize}

\noi
Done.  
\medskip 


\begin{itemize}
\item[] \textit{5, 35 (and elsewhere): Careful! This isn't quite the sample variance! This is actually the sample variance multiplied by $m/n$.}
\end{itemize}

\noi
Thanks for catching this. 
We corrected this sentence. 
Please note that we broke this sentence into two to avoid long sentences and to improve exposition. 
We also checked for other uses of the term `sample variance' throughout the paper and corrected these as needed.
Thank you.  
\medskip 


\begin{itemize}
\item[] \textit{5, 43: You could have introduced the notation ``CI'' earlier.}
\end{itemize}

\noi
You are right. 
We now introduce this abbreviation the first time we use the term `confidence interval' in the introduction. 
From this point on, we use the abbreviated form `CI' throughout the paper, except for the title of Section 4.5 (Asymptotic Validity of the Confidence Intervals). 
\medskip 



\begin{itemize}
\item[] \textit{5: You can significantly reduce the discussion on OBM here and in \S 3.1. 
Just let the reader know the main results that you'll use.}
\end{itemize}

\noi
We removed \S 3.1 (Variably Overlapping Batches) from the paper. 
We merged this section with \S 2.1.2 (Overlapping Batch Means), focusing on the results used in the paper. 
We believe the presentation of the paper improved as a result. 
Thank you for this suggestion. 
\medskip 



\begin{itemize}
\item[] \textit{6, 13: ``corresponding overlapping batch means''}
\end{itemize}

\noi
Done. 
\medskip 



\begin{itemize}
\item[] \textit{6, 29: Note that you use ``Note that'' too often.}
\end{itemize}

\noi
We went through the paper and removed most of the ``note that'' phrases. 
We decided to leave a couple as is (mainly as ``notice that'') because they highlight important characteristics.  
\medskip 



\begin{itemize}
\item[] \textit{6, 30: Explicitly say what the d.f.\ in (2) is.}
\end{itemize}

\noi
We changed this sentence and directly referred to the denominator term.
We decided to avoid the degrees of freedom here. 
This is because we wanted to focus on the degrees of freedom for confidence interval construction, which is discussed in both the theoretical and computational results.\smallskip  

\noi 
{\it Note:} This section has been restructured based on your comments. 
Equation (2) now gives the general overlapping variance estimator with the parametrized amount of overlap. 
\medskip 



\begin{itemize}
\item[] \textit{7, 46: Does $\gamma$ need to be an integer? 
You seem to be using it that way, so maybe state the requirement.}
\end{itemize}

\noi
Yes; we explicitly stated this in a sentence after we introduce $\gamma$.\smallskip 

\noi 
{\it Note:} This sentence now appears in Section 2.1.2 (Overlapping Batch Means). 
\medskip 



\begin{itemize}
\item[] \textit{8, 4 (and elsewhere): Define notations such as floor and ceiling functions.}
\end{itemize}

\noi
Done. 
%We introduced the floor function before its first use, and we introduced the ceiling function right after its first use.
\smallskip  

\noi 
{\it Note:} Both definitions appear in Section 2.1.2 (Overlapping Batch Means).
\medskip 



\begin{itemize}
\item[] \textit{8, 11: In addition to the fact that I don't think ``variance reduction'' is quite the right term, the phrase ``variance reduction in this estimator'' is awkward.}
\end{itemize}

\noi
Reworded.\smallskip 

\noi 
{\it Note:} This sentence now appears in Section 2.1.2 (Overlapping Batch Means). 
\medskip 



\begin{itemize}
\item[] \textit{8, 53--57: The various OBM references should have been discussed earlier.}
\end{itemize}

\noi
You are right. 
We moved these references---in particular, Damerdji (1994, 1995)---to Section 2.1.2 (Overlapping Batch Means) and added a brief discussion.  
\medskip 



\begin{itemize}
\item[] \textit{10, 11 (and elsewhere): ``$E[f(x, \xi)^4]$'' instead of ``$Ef(x, \xi)^4$'' (never hurts to use some brackets to reduce ambiguity)}
\end{itemize}

\noi
Changed.
In fact, we now use brackets for all the expectations in the paper to improve clarity. 
Thank you.
\medskip 



\begin{itemize}
\item[] \textit{11, 33: It might be helpful to add a step before the first inequality in the proof.}
\end{itemize}

\noi
Instead of adding another step before the first inequality in the proof of Lemma 1, we added explanations at the beginning of the proof to make this inequality clear.   
\medskip 



\begin{itemize}
\item[] \textit{12, 52: ``$n_l VD_l$ has been studied''}
\end{itemize}

\noi
Done.
\medskip 



\begin{itemize}
\item[] \textit{13, 12 (and elsewhere): I may have missed this, but has the notation $\sigma_{\hat{x}}$ been established yet?}
\end{itemize}

\noi
xxx 
This notation was defined before Theorem 1. 
We moved the definition of $\sigma_{\hat{x}}$ and $\mu_{\hat{x}}$ to \ldots to improve clarity. 
\medskip 



\begin{itemize}
\item[] \textit{In any case, some of the verbiage in this paragraph is a bit repetitive.}
\end{itemize}

\noi
We moved the discussion at the first half of this paragraph to the new, combined OBM review in \S 2.1.2 (Overlapping Batch Means). 
We referred to equation (3) here to remind the readers the basis of the proof.
The second half of this paragraph deals with the batching structure, and this discussion is moved to the paragraph after the theorem's proof. 
\medskip 



\begin{itemize}
\item[] \textit{13, 44: Is the expression for d.f. too big?}
\end{itemize}

\noi
We now denote it as `dt' to represent the whole denominator term. 
\medskip 



\begin{itemize}
\item[] \textit{13, 53: The reader might need another step besides C-S to get that inequality.}
\end{itemize}

\noi
This is a direct application of the Cauchy-Schwartz inequality $\left(\sum_{j=1}^{n}x_j y_j\right)^2 \leq \sum_{j=1}^{n}x_j^2 \sum_{j=1}^{n}y_j^2$, where the $y_j$ are all set to $1$. 
We added a brief explanation after this inequality to make it clear. 
\medskip 



\begin{itemize}
\item[] \textit{15, 25: Do you discuss $t_{d_l,\alpha}$ at all in Theorem 3?}
\end{itemize}

\noi
No, we did not.   
We added explanations in the proof Theorem 3, discussing how $t_{d_l,\alpha}$ factors in the proof. 
\medskip 



\begin{itemize}
\item[] \textit{18: Your figures are very difficult to read, even with the pdf file.}
\end{itemize}

\noi
We split these two figures into three and enlarged each graph. 
We also enlarged the legends in these graphs to improve readability.
All the new graphs added to the paper in Section 5.4 (Computational Efficiency) are also larger, with legible legends. 
\medskip 



\begin{itemize}
\item[] \textit{20, 47: ``4th edition''}
\end{itemize}

\noi
Done.
\medskip 



\begin{itemize}
\item[] \textit{21, 23: ``\it Jackknife''}
\end{itemize}

\noi
Corrected. Thank you. 


\end{document}
