\documentclass[11pt,notitlepage,onecolumn]{article}
%%%%%%%%%%%%%%%%%%%%%%%%%%%%%%%%%%%%%%%%%%%%%%%%%%%%%%%%%%%%%%%%%%%%%%%%%%%%%%%%%%%%%%%%%%%%%%%%%%%%%%%%%%%%%%%%%%%%%%%%%%%%%%%%%%%%%%%%%%%%%%%%%%%%%%%%%%%%%%%%%%%%%%%%%%%%%%%%%%%%%%%%%%%%%%%%%%%%%%%%%%%%%%%%%%%%%%%%%%%%%%%%%%%%%%%%%%%%%%%%%%%%%%%%%%%%
\usepackage{makeidx}
\usepackage{amsfonts}
\usepackage{amsmath}
\usepackage{amssymb}
\usepackage{chicago}
\usepackage[singlespacing]{setspace}

\setcounter{MaxMatrixCols}{10}

\setlength{\textheight}{9in}
\setlength{\textwidth}{6.5in}
\setlength{\topmargin}{-0.5in}
\setlength{\oddsidemargin}{0.0in}

\newcommand{\noi}{\noindent}

\begin{document}

%%%%%%%%%%%%%%%%%%%%%%%%%%%%%%%%%%%%%%%%%%%%%%%%%%%%%%%%%%%%%%%%%%%%%%%%%%%

\singlespacing

\baselineskip0.26in

%%%%%%%%%%%%%%%%%%%%%%%%%%%%%%%%%%%%%%%%%%%%%%%%%%%%%%%%%%%%%%%%%%%%%%%%%%%

\pagebreak

\
\begin{center}
\textbf{\Large Response to Referee 2} 
\medskip

{\large Overlapping Batches for the Assessment of Solution Quality in Stochastic Programs}
\medskip

{\footnotesize by D. Love and G. Bayraksan}
\end{center}

\bigskip


%%%%%%%%%%%%%%%%%%%%%%%%%%%%%%%%%%%%%%%%%%%%%%%%%%%%%%%%%%%%%%%%%%%%%%%%%%%

We thank the referee for the helpful comments and suggestions. 
We incorporated all of the recommended changes and believe the paper is improved as a result. 
Below, we go through in detail these changes. 
\medskip

\bigskip 

%%%%%%%%%%%%%%%%%
%% Need to be careful with the minor changes. Will these be affected?
%% Need to CHECK AT THE VERY END... FOR ANY CHANGES on TOP OF these...
%%    - GB
%%%%%%%%%%%%%%%%%

\noi  
{\large \bf General Comments:}
\medskip 


\begin{itemize}
\item[1.] \textit{The main contributions of the paper need to be stated more clearly and succinctly.}
\end{itemize}

\noindent 
xxx  We agree. 
We changed the introduction to clearly state the contributions of the paper and \ldots 
\medskip 


\begin{itemize}
\item[2.] \textit{It is not necessary to go into such detail in the OBM review.}
\end{itemize}

\noindent 
xxx  We shortened the section on OBM review by combining \ldots 
\medskip 

\begin{itemize}
\item[3.] \textit{The English needs to be tightened up somewhat.}
\end{itemize}

\noindent  
xxxx
You are right. 
We went through the paper and tightened up the prose. 
We also incorporated all the changes you suggested below and hope the exposition of the paper is improved as a result.
We hope that any remaining English issues could be corrected by the English editor of the journal. (IS THERE ONE????) 
xxxx
\medskip 

\bigskip 


\noi  
{\large \bf Specific Comments:}
\medskip 


\begin{itemize}
\item[] \textit{Page 1, Line 32 (and elsewhere): I believe that you ``apply the OBM method to stochastic programming'' rather than extend it.}
\end{itemize}

\noi
Changed.  xxx Also go through to see others! 
\medskip 



\begin{itemize}
\item[] \textit{1, 35 (and elsewhere): ``We give conditions'' instead of ``We show conditions''}
\end{itemize}

\noi
Changed. xxx Also go through to see others!
\medskip 


\begin{itemize}
\item[] \textit{1, 37: What are the classical results? Reference?}
\end{itemize}

\noi
We slightly reworded this sentence and provided references.   
\medskip 



\begin{itemize}
\item[] \textit{1, 49: Suggested rewrite to reduce ambiguity: ``The overlapping batch means (OBM) \ldots to obtain variance estimators having variances that are comparatively lower than their nonoverlapped counterparts. 
In the context of steady-state simulation, OBM is often used to estimate the variance of the sample mean, which is itself an estimator for the mean. 
Schmeiser et al.\ [1990] use overlapping estimators to estimate the
variance of estimators for non-means. 
More recently, overlapping estimators based on standardized time series (as opposed to batch means) have been evaluated [Alexopoulos et al., 2007a,b].'' 
In this rewrite, note the use of active and passive referencing formats.}
\end{itemize}

\noi
Thank you for this suggestion of rewrite and for your careful review of the paper. 
We rewrote this paragraph based on your corrections.
We also went through the paper and made changes to the language, paying specific attention to active and passive referencing formats. 
\medskip 


\begin{itemize}
\item[] \textit{2, 5: ``never before been studied . . . context.''}
\end{itemize}

\noi
Changed.
\medskip 


\begin{itemize}
\item[] \textit{2, 7: By ``same benefits'', you mean moderate bias and comparatively low variance, right?}
\end{itemize}

\noi
xxx  
\medskip 


\begin{itemize}
\item[] \textit{2, 8: Start a new paragraph with ``A natural setting\ldots ''.}
\end{itemize}

\noi
Done.
\medskip 


\begin{itemize}
\item[] \textit{2, 10: ``especially'' instead of ``esp.''. 
Also, ``This'' instead of ``Assessing solution quality in stochastic programming'' (since it's repetitive)}
\end{itemize}

\noi
Both done.  
\medskip 


\begin{itemize}
\item[] \textit{2, 13: ``Because optimality gap''}
\end{itemize}

\noi
Done.
\medskip 


\begin{itemize}
\item[] \textit{2, 24: ``under which the resulting gap estimators exhibit reduced variance when compared to their counterparts obtained via nonoverlapping batches.''}
\end{itemize}

\noi
Done.
\medskip 


\begin{itemize}
\item[] \textit{2, 24 (and elsewhere): Note: I would be careful about the use of the term ``variance reduction'', which usually refers to the class of techniques such as common random numbers, antithetics, importance sampling, etc. 
I view what you're doing in this paper as simply proposing a new estimator whose variance is lower than that of a competing estimator.}
\end{itemize}

\noi
xxx  You are right. 
Thanks for pointing this out. We corrected the wording throughout the paper. 
\medskip 


\begin{itemize}
\item[] \textit{2, 27: ``and that the interval \ldots valid. 
We also present''}
\end{itemize}

\noi
xxx  
\medskip 


\begin{itemize}
\item[] \textit{2, 28: ``Our experiments indicate that the use of overlapping estimators results in gap estimators achieving lower variance than their nonoverlapping counterparts, along with comparable bias and interval coverage --- for both small- and large-sample cases.}
\end{itemize}

\noi
xxx  
\medskip 


\begin{itemize}
\item[] \textit{2, 34 (and elsewhere): Might want to boldface all vectors.}
\end{itemize}

\noi
xxx  
\medskip 


\begin{itemize}
\item[] \textit{2, 42 (and elsewhere): ``more-restrictive moment'' (hyphenate certain non-adverbial compound adjectives)}
\end{itemize}

\noi
xxx  Thank you; we went through the paper and hyphenated all the compound adjectives.
\medskip 


\begin{itemize}
\item[] \textit{2, 51: You can delete ``(SAA)'', since the notation isn't used again in the paper.}
\end{itemize}

\noi
xxx  
\medskip 


\begin{itemize}
\item[] \textit{3, 11: ``solution; and''}
\end{itemize}

\noi
xxx  
\medskip 


\begin{itemize}
\item[] \textit{3, 24 (and elsewhere): ``Lagrangian, or'' (comma after penultimate entry in a conjunctive list)}
\end{itemize}

\noi
xxx  
\medskip 


\begin{itemize}
\item[] \textit{3, 37: ``we can simply use $z_m^*$''}
\end{itemize}

\noi
xxx  
\medskip 


\begin{itemize}
\item[] \textit{3, 38: ``estimator of the optimality''}
\end{itemize}

\noi
xxx  
\medskip 


\begin{itemize}
\item[] \textit{3, 53: ``The sample variance of these estimators is used to form confidence intervals for the optimality gap (we further review MRP in \S 2.2).''}
\end{itemize}

\noi
xxx  
\medskip 


\begin{itemize}
\item[] \textit{4, 5 (and elsewhere): ``see, e.g., Bertocchi et al. [2000]; \ldots ''}
\end{itemize}

\noi
xxx  I don't think I agree with this. 
These need to be in parentheses! 
\medskip 


\begin{itemize}
\item[] \textit{4, 14: Delete the sentence ``Especially \ldots '' since it doesn't really add anything to the discussion.}
\end{itemize}

\noi
xxx  Deleted.
\medskip 


\begin{itemize}
\item[] \textit{4, 17: ``interest is that of estimating''}
\end{itemize}

\noi
xxx  
\medskip 


\begin{itemize}
\item[] \textit{4, 20--27: Even though I believe everything, these lines are a bit awkward and confusing.}
\end{itemize}

\noi
xxx   DAVID-- NEED YOUR HELP with this. 
I don't know how to reword it to make it less awkward.
\medskip 


\begin{itemize}
\item[] \textit{4, 31: What are the desired statistical properties?}
\end{itemize}

\noi
xxx  
\medskip 


\begin{itemize}
\item[] \textit{4, 33: ``Another difference is that''. By the way, what is ``difference'' referring to?}
\end{itemize}

\noi
xxx  
\medskip 


\begin{itemize}
\item[] \textit{4, 36: ``due to the need to solve sampling problem"}
\end{itemize}

\noi
xxx  
\medskip 


\begin{itemize}
\item[] \textit{4, 42: ``Partial overlapping results in a smaller number of batches; hence, fewer optimization''}
\end{itemize}

\noi
xxx  
\medskip 


\begin{itemize}
\item[] \textit{4, 45: The sentence ``Second, \ldots '' needs some clarification, since it states a conclusion that hasn't so far been justified.}
\end{itemize}

\noi
xxx  
\medskip 


\begin{itemize}
\item[] \textit{4, 51: ``discuss overlapping''}
\end{itemize}

\noi
xxx  
\medskip 


\begin{itemize}
\item[] \textit{4, 54 (and elsewhere): When you refer to ``variable amounts of overlap'', that is really nothing new. Certainly, any overlapping estimator will have end effects in which the overlap amounts differ from those in the ``middle'' region of the data.}
\end{itemize}

\noi
xxx  
\medskip 



\begin{itemize}
\item[] \textit{4, 57 (and elsewhere): ``classical case'' refers to nonoverlapping batch means, right?}
\end{itemize}

\noi
xxx  Yes. 
We reworded this sentence to make it clear.
\medskip 



\begin{itemize}
\item[] \textit{5, 12: Does $\sigma^2$ refer to the marginal variance or the ``variance parameter'' (sum of the covariances at all lags)?}
\end{itemize}

\noi
xxx   CHECK -- DON'T REMEMBER
\medskip 



\begin{itemize}
\item[] \textit{5, 14: ``realization of the univariate stochastic''}
\end{itemize}

\noi
xxx  
\medskip 


\begin{itemize}
\item[] \textit{5, 35 (and elsewhere): Careful! This isn't quite the sample variance! This is actually the sample variance multiplied by $m/n$.}
\end{itemize}

\noi
xxx  Thanks for catching this! We corrected this and others.
\medskip 


\begin{itemize}
\item[] \textit{5, 43: You could have introduced the notation ``CI'' earlier.}
\end{itemize}

\noi
xxx  You are right. 
We now do introduce it earlier in the paper -- GIVE PAGE NO.
\medskip 



\begin{itemize}
\item[] \textit{5: You can significantly reduce the discussion on OBM here and in \S 3.1. 
Just let the reader know the main results that you'll use.}
\end{itemize}

\noi
xxx  
\medskip 



\begin{itemize}
\item[] \textit{6, 13: ``corresponding overlapping batch means''}
\end{itemize}

\noi
xxx  
\medskip 



\begin{itemize}
\item[] \textit{6, 29: Note that you use ``Note that'' too often.}
\end{itemize}

\noi
xxx  
\medskip 



\begin{itemize}
\item[] \textit{6, 30: Explicitly say what the d.f.\ in (2) is.}
\end{itemize}

\noi
xxx  
\medskip 



\begin{itemize}
\item[] \textit{7, 46: Does $\gamma$ need to be an integer? 
You seem to be using it that way, so maybe state the requirement.}
\end{itemize}

\noi
xxx  CHECK THIS
\medskip 



\begin{itemize}
\item[] \textit{8, 4 (and elsewhere): Define notations such as floor and ceiling functions.}
\end{itemize}

\noi
xxx  
\medskip 



\begin{itemize}
\item[] \textit{8, 11: In addition to the fact that I don't think ``variance reduction'' is quite the right term, the phrase ``variance reduction in this estimator'' is awkward.}
\end{itemize}

\noi
xxx  
\medskip 



\begin{itemize}
\item[] \textit{8, 53--57: The various OBM references should have been discussed earlier.}
\end{itemize}

\noi
xxx  
\medskip 



\begin{itemize}
\item[] \textit{10, 11 (and elsewhere): ``$E[f(x, \xi)^4]$'' instead of ``$Ef(x, \xi)^4$'' (never hurts to use some brackets to reduce ambiguity)}
\end{itemize}

\noi
xxx Changed. 
\medskip 



\begin{itemize}
\item[] \textit{11, 33: It might be helpful to add a step before the first inequality in the proof.}
\end{itemize}

\noi
xxx  
\medskip 



\begin{itemize}
\item[] \textit{12, 52: ``$n_l VD_l$' has been studied''}
\end{itemize}

\noi
xxx  
\medskip 



\begin{itemize}
\item[] \textit{13, 12 (and elsewhere): I may have missed this, but has the notation $\sigma_{\hat{x}}$ been established yet? 
In any case, some of the verbiage in this paragraph is a bit repetitive.}
\end{itemize}

\noi
xxx  
\medskip 



\begin{itemize}
\item[] \textit{13, 44: Is the expression for d.f. too big?}
\end{itemize}

\noi
xxx  
\medskip 



\begin{itemize}
\item[] \textit{13, 53: The reader might need another step besides C-S to get that inequality.}
\end{itemize}

\noi
xxx  
\medskip 



\begin{itemize}
\item[] \textit{15, 25: Do you discuss $t_{d_l,\alpha}$ at all in Theorem 3?}
\end{itemize}

\noi
xxx  CHECK. DON'T RECALL.
\medskip 



\begin{itemize}
\item[] \textit{18: Your figures are very difficult to read, even with the pdf file.}
\end{itemize}

\noi
xxx  
\medskip 



\begin{itemize}
\item[] \textit{20, 47: ``4th edition''}
\end{itemize}

\noi
Done.
\medskip 



\begin{itemize}
\item[] \textit{21, 23: ``\it Jackknife''}
\end{itemize}

\noi
Corrected. Thank you. 


\end{document}
